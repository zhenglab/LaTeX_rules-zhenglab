\documentclass[a4paper,12pt]{article}
\usepackage{styles/iplouccfg}
\usepackage{styles/zhfontcfg}



\title{实验室工作说明} %标题
\author{CVBIOUC}
\date{2014年9月5日} %日期,不加默认为当前日期

\begin{document}

\maketitle

\section{安全与健康}

\begin{enumerate}
\item 无论什么时候什么情况下,安全与健康永远都是第一位的!
\item 我们崇尚快乐科研,高效工作!
\item 优秀离不开坚持不懈专注用心的努力,但安全和健康要比任何事都重要,以下要求请务必在保证安全和身体健康的前提下进行!
\end{enumerate}

\section{工作时间}

\begin{description}
\item[上午] 8:30--11:30
\item[下午] 2:00-5:00
\item[晚上] 时间自定
\end{description}

\begin{itemize}
\item 请确保每天工作8个小时、每周至少工作5天,以保证每周工作时间能在{\color{red}{40小时}}以上。
\item 实验室暑假10天左右,寒假在一个月以内。
\item 如果超过半天(包括半天)不能到实验室工作或每周工作时间达不到40小时(有课时,上课时间按工作时间计算),须与导师进行沟通说明。
\item 但无论怎样,都不能拿出休息的时间来工作;当你觉得你不得不这样做的时候,你应该首先考虑自己是否做到了合理高效的利用时间。
\end{itemize}

\section{工作要求}

\begin{itemize}
\item 实验室不允许做与学习、科研等无关的事情,尤其浏览无关的网站和视频。
\item 以坚持、专注、用心、积极、进取的态度全身心的投入到科研工作中。
\end{itemize}

\section{科研要求与国内外学术交流}

\begin{itemize}
\item 对团队有高度的责任心,以开放、诚恳、学习的心态与团队成员交流、学习。
\item 有任何想法或问题都要及时跟老师和团队进行沟通。
\item 目前实验室在未来 3 年有比较充足的资金支持大家从事科研、交流(国内和国际研讨与会议)、发表等工作。
\item 鼓励向计算机视觉相关国际顶级会议投稿并到国外参加会议!
\item 一个合格的硕士研究生在毕业的时候应该有一篇国际会议论文和一篇国际期刊论文。
\end{itemize}

目前我们主要关注的计算机视觉国际会议:

\begin{itemize}
\item 一级会议:ICCV、CVPR、ECCV
\item 二级会议:ICIP、ICPR、ACCV
\end{itemize}
\section{科研安排}

以下工作须在与导师不断的沟通下完成:

\begin{enumerate}
\item 研究生一年级第一学期的前两个月:熟悉必要的科研工具,学习必要的基础知识(图像处理相关),进行必要的科研训练。查阅方向相关的科研书籍与学术论文,明确各自的科研方向(大)。
\item 研究生一年级第一学期的后两个月:根据各自的科研方向,结合实验室的科研项目,寻找具体可行的科研方向(小且细) 。在研究生一年级学期结束前,每人务必完成一篇关于自己研究方向的综述。
\item 研究生一年级第二学期(包括研一第二学期末暑假):根据所明确的小且细的科研方向,不断实验(实践)、反复思考、频繁讨论,进行深入的科研工作,撰写相关学术论文并投稿(国际会议)。
\item 研究生二年级第一学期、第二学期(包括研二第二学期末暑假):根据前期工作进一步完善,进一步撰写相关学术论文并投稿(国际期刊)。
\item 研究生三年级第一学期、第二学期:根据前面两年的工作积累,构思硕士毕业论文,并进行必要的工作补充。并进行硕士毕业论文的撰写、答辩等工作。
\end{enumerate}

\section{科研工具}

\begin{description}
\item[操作系统] Linux/Mac OS
\item[编程语言] Matlab,C/C++(OpenCV)与Python
\item[文档排版工具] \LaTeX
\item[版本控制系统] Git/GitHub
\item[文献管理软件] BibDesk等
\end{description}

\section{邮件列表组(Google Groups)}

\begin{itemize}
\item 组名:cvbiouc@googlegroups.com
\item 用途:主要用于日常交流、沟通、讨论以及通知、消息发布和各种分享等。
\item 使用:发邮件给cvbiouc@googlegroups.com;如果需要对同一个话题进行讨论直接进行邮件“回复”(请尽量不要新开邮件除非你需要开启新话题的讨论)。
\item 注意:如新加或更换个人邮箱请及时与郑老师说明替换。
\end{itemize}

\section{公共盘}

\begin{itemize}
\item 地址:222.195.149.61
\item 用途:主要存放实验室各种实验数据(图像等),以及历年毕业的博士和硕士资料(毕业论文、发表论文、程序数据、文章报告等),同时也可用于自己的资料备份。
\item 说明:samba服务器一共四个虚拟磁盘:data、software、research、study。其中data主要存放数据,software主要存放软件,research主要存放项目和学习资料,study主要存放历年毕业的博硕士资料及各种视频资料。study磁盘的使用频率最高。
\item 使用:采用samba配置,各种操作系统下均可通过驱动器映射的方式将公共盘挂载到本机使用,但为了安全考虑需要[用户名+密码+ip地址]绑定的方式才可以登录。个人存放资料的地址在study盘下MS(硕士)或PhD(博士)文件夹下相应毕业年份目录下标有自己名字的文件夹。其中硕士的个人目录(相当于你在公共盘上的HOME)下已经建好分类文件夹,请平时注意根据个人情况整理备份,如有特殊需求请跟郑老师沟通。
\item 注意:如新加或更换个人电脑ip地址请及时跟郑老师说明替换(使用wifi上网者不需要)。
\item 用户名和密码:新加入者默认都是自己姓名的全拼。但由于此项服务没有设定自己修改密码的功能,所以新加入者请务必在测试可用后单独告知郑老师想要设置的密码来进行修改。
\end{itemize}

\section{个人主页(HomePage)}

\begin{itemize}
\item 地址:211.64.142.66
\item 用途:用于发布个人主页。
\item 使用:ssh xingming@211.64.142.66 远程登录后在个人目录HOME下的public目录下便是对外发布的http://vision.ouc.edu.cn/$\sim$xingming 地址。
\item 注意:未绑定IP地址访问时请配置好ssh的代理服务器设置。
\item 用户名和密码:新加入者默认都是自己姓名的全拼,请及时登录后使用passwd命令修改个人密码。
\item 参考:\url{http://vision.ouc.edu.cn/~zhenghaiyong}、 \url{http://vision.ouc.edu.cn/~sunxue}、 \url{http://vision.ouc.edu.cn/~zhaohongmiao}
\end{itemize}

\section{MOOC学习}

\begin{itemize}
\item 研究生一年级同学除了完成学校要求的研究生培养计划外,还必须完成实验室指定课程(MOOC课程或“慕课”)的学习。
\item 由于网上MOOC课程可能会有所更新和变更等,所以每学年指定课程可能不尽相同,但总体来说会包括科研、程序、专业三部分。
\item 2014级必修课程包括:

\begin{enumerate}
\item 中国科学技术大学《文献管理与信息分析》。其中最新的文献管理课程还未开课,可以先学习网上最近一期的存档视频(\url{http://www.icourse163.org/learn/ustc-9002?tid=9002#/learn/announce})。
对于课程具体内容的选择,可以根据科研需要、自己的兴趣而定,如有疑问,请及时与郑老师沟通。
\item 北京大学《计算概论A》:\url{https://www.coursera.org/course/pkuic}
\end{enumerate}

\item 要求:

\begin{enumerate}
\item 每周公开课学习后请及时下载相关视频并上传公共盘以供大家学习。
\item 每周公开课学习后请务必使用英文向全组【邮件列表组地址:\href{mailto:zhenglabouc@googlegroups.com}{cvbiouc@googlegroups.com}】通过邮件发表个人总结、观点、收获、感想等。
\end{enumerate}

\end{itemize}

\section{英语学习}

\begin{description}
\item[读] 除了必要的英文论文阅读外,请每天拿出一定时间(半个小时到一个小时)来阅读感兴趣的英语杂志等。
\item[写] 必须通过邮件、文档等尽力进行英文写作训练,只读不写无法有真正的提高。
\item[听] 在非工作时间可以讨论观看励志、科技等方面的英文视频。
\item[说] 每位研究生在就读期间争取能够到国外参加一次国际会议,以锻炼与人英语沟通的能力。
\end{description}

\section{工作环境}

\begin{itemize}
\item 在经费允许的情况下,实验室会为每位研究生配备科研相关的环境,如计算机、办公用品、打印复印等。
\item 在经费允许的情况下,根据工作情况发放劳务费。
\item 对于研究生期间的优秀成果(高质量会议和期刊论文),实验室会进行不同程度的奖励和鼓励。
\item 其他必要的要求和建议请及时沟通,在必要的情况下实验室会尽力解决。
\end{itemize}

\end{document}
