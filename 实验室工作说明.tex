\documentclass[a4paper,12pt]{article}
\usepackage{styles/iplouccfg}
\usepackage{styles/zhfontcfg}



\title{实验室工作说明} %标题
\author{郑海永}
\date{2013年10月21日} %日期,不加默认为当前日期

\begin{document}

\maketitle

\section{安全与健康}

\begin{enumerate}
\item 无论什么时候什么情况下,安全与健康永远都是第一位的!
\item 我们崇尚快乐科研,高效工作!
\item 优秀离不开坚持不懈专注用心的努力,但安全和健康要比任何事都重要,以下要求请务必在保证安全和身体健康的前提下进行!
\end{enumerate}

\section{工作时间}

\begin{description}

\item[上午] 8:30--11:30

\item[下午] 2:00-5:00

\item[晚上] 6:30--9:30

\end{description}

\begin{itemize}

\item 请确保每天工作9个小时、每周至少工作6天,以保证每周工作时间能在{\color{red}{50小时}}以上。

\item 实验室没有暑假,寒假在一个月以内。

\item 如果超过半天(包括半天)不能到实验室工作或每周工作时间达不到50小时(有课时,上课时间按工作时间计算),须与导师进行沟通说明。

\item 但无论怎样,都不能拿出休息的时间来工作;当你觉得你不得不这样做的时候,你应该首先考虑自己是否做到了合理高效的利用时间。

\end{itemize}

\section{工作要求}

\begin{itemize}

\item 实验室不允许做与学习、科研等无关的事情,尤其浏览无关的网站和视频。

\item 以坚持、专注、用心、积极、进取的态度全身心的投入到科研工作中。

\end{itemize}

\section{科研要求与国内外学术交流}

\begin{itemize}

\item 对团队有高度的责任心,以开放、诚恳、学习的心态与团队成员交流、学习。

\item 有任何想法或问题都要及时跟老师和团队进行沟通。

\item 目前实验室在未来 3 年有比较充足的资金支持大家从事科研、交流(国内和国际研讨与会议)、发表等工作。

\item 鼓励向计算机视觉相关国际顶级会议投稿并到国外参加会议!

\item 一个合格的硕士研究生在毕业的时候应该有一篇国际会议论文和一篇国际期刊论文。

\end{itemize}

目前我们主要关注的计算机视觉国际会议:

\begin{itemize}

\item 一级会议:ICCV、CVPR、ECCV

\item 二级会议:ICIP、ICPR、ACCV

\end{itemize}

\section{科研安排}

以下工作须在与导师不断的沟通下完成:

\begin{enumerate}

\item 研究生一年级第一学期的前两个月:熟悉必要的科研工具,学习必要的基础知识(图像处理相关),进行必要的科研训练。

\item 研究生一年级第一学期的后两个月;查阅方向相关的科研书籍与学术论文,明确各自的科研方向(大)。

\item 研究生一年级第二学期:根据各自的科研方向,结合实验室的科研项目,寻找具体可行的科研方向(小且细)
。

\item 研究生二年级第一学期(包括研一第二学期末暑假):根据所明确的小且细的科研方向,不断实验(实践)、反复思考、频繁讨论,进行深入的科研工作,撰写相关学术论文并投稿(国际会议)。

\item 研究生二年级第二学期(包括研二第二学期末暑假):根据前期工作进一步完善,进一步撰写相关学术论文并投稿(国际期刊)。

\item 研究生三年级第一学期:根据前面两年的工作积累,构思硕士毕业论文,并进行必要的工作补充。

\item 研究生三年级第二学期:硕士毕业论文的撰写、答辩等工作。

\end{enumerate}

\section{科研工具}

\begin{description}

\item[操作系统] UNIX/Linux

\item[编程语言] Matlab与C/C++(OpenCV)

\item[文档排版工具] \LaTeX

\item[版本控制系统] Git/GitHub

\end{description}

\section{MOOC学习}

\begin{itemize}

\item 研究生一年级同学除了完成学校要求的研究生培养计划外,还必须完成实验室指定课程(MOOC课程或“慕课”)的学习并务必在一年级结束前获得学分证书后方可最后毕业。

\item 由于网上MOOC课程可能会有所更新和变更等,所以每学年指定课程可能不尽相同,但总体来说会包括科研、程序、数学和专业四部分,其中数学和专业对于不同研究方向的同学会略有所不同【目前实验室研究方向包括计算机视觉和生物信息学,这两个方向在数学和专业上的要求也略有不同】。

\item 2013级和2014级(保研)必修课程包括:

\begin{enumerate}

\item 中国科学技术大学《文献管理与信息分析》:\url{http://mooc.ustc.edu.cn}
\item 北京大学《计算概论A》:\url{https://www.coursera.org/course/pkuic}
\item 计算机视觉方向加必修《Numerical Analysis》和《Mathematical Problems in Image Processing》【自学+每周汇报与讨论的方式】
\item 生物信息学方向加必修《鸟哥的Linux私房菜:基础学习篇和服务器架设篇》(第三版)【自学+每周汇报与讨论的方式】和北京大学《生物信息学:导论与方法》:\url{https://www.coursera.org/course/pkubioinfo}

\end{enumerate}

\item 要求:

\end{itemize}

\begin{enumerate}

\item 每周公开课学习后请务必使用英文向全组【邮件列表组地址:\href{mailto:zhenglabouc@googlegroups.com}{zhenglabouc@googlegroups.com}】通过邮件发表个人总结、观点、收获、感想等。
\item 按照每门课程的要求完成课程学习并获得相应的学分证书,如果第一次未通过或错过选课时间等,请下学期继续注册学习直至获得学分证书,并务必在研究生一年级结束前修得学分获得证书。

\end{enumerate}

\section{英语学习}

\begin{description}

\item[读] 除了必要的英文论文阅读外,请每天拿出一定时间(半个小时到一个小时)来阅读感兴趣的英语杂志等。

\item[写] 必须通过邮件、文档等尽力进行英文写作训练,只读不写无法有真正的提高。

\item[听] 在非工作时间可以讨论观看励志、科技等方面的英文视频。

\item[说] 每位研究生在就读期间争取能够到国外参加一次国际会议,以锻炼与人英语沟通的能力。

\end{description}

\section{公共盘}

\begin{itemize}

\item 公共盘为实验室历年积累的资料,主要包括实验室数据、毕业研究生
的论文和程序等,还有一些经典的视频资料,供大家参考。

\item 新进学生需提供用户名和密码。

\end{itemize}

\section{工作环境}

\begin{itemize}

\item 在经费允许的情况下,实验室会为每位研究生配备科研相关的环境,如计算机、办公用品、打印复印等。

\item  在经费允许的情况下,根据工作情况发放劳务费。

\item 对于研究生期间的优秀成果(高质量会议和期刊论文),实验室会进行不同程度的奖励和鼓励。

\item 其他必要的要求和建议请及时沟通,在必要的情况下实验室会尽力解决。

\end{itemize}

\end{document}
